\documentclass[10pt]{article}
\usepackage[a4paper,margin=1in,footskip=0.25in]{geometry}
\usepackage[final]{microtype}
\usepackage{fancyvrb}
\usepackage{xcolor}
\usepackage{amsmath}
\usepackage{pythonhighlight}

\title{\vspace{-3cm}Hvieznde impérium}

\vspace{0pt}
\begin{document}
\author{}
\date{}
\maketitle

\section*{Kružnica}

Ak je impérium kružnica, každý systém musí mať iba susedný systém, ktorý je na
kružnici za ním a systém, ktorý je na kružnici pred ním. V kružnici musí byť
tak každý systém predchodca aj nasledovateľ nejakého iného systému. Ak bude mať
každý systém stupeň $2$, teda je na kružnici predchodca aj nasledovateľ, vieme
povedať, že impérium je kružnica. Z toho vyplíva, že ak nejaký systém
nemá dvoch susedov, impérium kružnica nie je, pretože impérium buď niekde
končí, alebo nejaký zo systémov má viac ako dvoch susedov a to nemôže nastať. 

Keďže neveštíme žiadne bity, pre toto riešenie je $k=0$.

\begin{python}
# ak pocet susedov nejakeho zo systemov nie je 2, imperium nie je kruznica
if stupen != 2: return NIE

# pockame a ak nikto neodpovedal nie, teda vesmir nie je ruzovy, odpoved je ANO
if ruzovy_vesmir(): return NIE
return ANO
\end{python}

\section*{Perfektné párovanie}

Vieme, že perfektne parovanie môže existovať iba v grafe s párnym počtom
vrcholov. Ak teda v impériu nie je počet systémov párny, rovno odpovedáme NIE.

Najprv si každý systém vyveští index jeho páru, ktorý následne pošle všetkym
jeho susediacim systémom. Ak systém po roku dostane od práve jedného susedného 
systému rovnaký index ako má on, neodpovedá NIE. V opačnom prípade odpovedá
NIE, keďže buď nemá pár, alebo sú v jeho páre viac ako $2$ systémy a to nemôže
nastať, keďže v páre môžu byť len $2$ systémy.

Párovanie existuje len vtedy, ak je možné nájsť každému systému práve jedného
suseda s rovnakým párom ako má daný systém. Indexy párov pre každý systém
nám vyveštia veštci, takže ak je možné túto podmienku splniť, vždy ju splnia,
pretože ak by vyveštili nejakému systému nesprávny index, zistili by sme to
vďaka vyvešteným indexom susedov daného impéria. Ak by nejaký systém mal
nesprávny index, teda by napríklad ani jeden jeho sused nemal rovnaký index,
tento systém by odpovedal NIE a celková odpoveď by kvôli tomu bola tiež NIE.
Ak by sa to nestalo a nikto by neodpovedal NIE, celková odpoveď by bola ANO.

Pre tento algoritmus používame $k=\lceil log_2 (n / 2) \rceil$, pretože veštíme
index páru pre každý systém a párov môže byť maximálne $n/2$ (podiel je vždy
celé číslo, keďže ak počet systémov je nepárny, párovanie neexistuje).

\begin{python}
# ak je pocet systemov neparny, prefektne parovanie neexistuje
if N % 2 != 0: return NIE
# vyvestime si index nasej dvojice parovania
p = vyvesti_cislo(N // 2)
if p >= N // 2: return NIE

# posleme nas vyvesteny index vsetkym nasim susedom
for i in range(stupen):
    outbox[i] = p

# zistime ci presne jeden nas sused je s nami v dvojici
pocet = 0
for i in range(stupen):
    if inbox[i] == p:
        pocet += 1

# ak nie je presne jeden nas susedny system s nami v dvojici, odpoved je NIE
if pocet != 1: return NIE

# pockame kym bude vesmir ruzovy, ak niekto mal odpoved NIE, perfektne
# parovanie neexistuje, inak mozeme povedat ze odpoved je ANO
if ruzovy_vesmir(): return NIE
return ANO
\end{python}

\section*{Hamiltonovská kružnica}

Pre Hamiltonovskú kružnicu platí to isté čo pre normálnu kružnicu, akurát je
možné, aby bol systém susedný aj s inými systémami okrem jeho predchodcu a 
následovateľa. Ak teda každý systém bude mať predchodcu a nasledovateľa na
kružnici, impérium je Hamiltonovská kružnica.

Na začiatku si každý systém vyveští svoj index na kružnici, ktorý následne
pošle všetkým svojim susedom. Systém po roku tak tiež dostane indexy všetkých 
svojich susedov. Ak medzi nimi bude index o jedna väčší ako index daného
systému, neodpovie NIE. V opačnom prípade odpovie NIE, pretože na kružnici
tento systém nemá nasledovateľa. Stačí nám zistiť, že každý systém je
nasledovateľ, pretože ak je každý nasledovateľ, každý je aj predchodca.

Ak je impérium Hamiltonovská kružnica, každému systému bude vyveštený index
tak, aby bol o $1$ menší ako index jeho nasledovateľa na kružnici. Ak to platí
pre všetky systémy, nikto neodpovie NIE a vieme, že impérium je Hamiltonovská
kružnica. Ak impérium Hamiltonovská kružnica nie je, jeden zo systémov bude
musieť mať za susedov iba systémy, ktoré majú iných predchodcov ako je daný
systém a teda ich index nie je o $1$ väčší ako index daného systému. Tento
systém odpovie NIE a tak aj celková odpoveď bude NIE.

Pre tento algoritmus je $k=\lceil log_2 n \rceil$, pretože každý systém si
vyveští jeho index na kružnici ktorý môže byť od $0$ po $n-1$.

\begin{python}
# ak ma nejaky system menej ako 2 susedov, imperium nie je kruznica, teda
# nemoze byt ani hamiltonovska kruznica
if stupen < 2: return NIE

# vyvestime si nas index na kruznici
p = vyvesti_cislo(N)
if p >= N: return NIE

# posleme nas vyvesteny index vsetkym nasim susedom
for i in range(stupen):
    outbox[i] = p

# zistime ci je jeden z nasich susedov nas nasledovatel na kruznici
nasledovatel = False
for i in range(stupen):
    if inbox[i] == (p + 1) % N:
        nasledovatel = True

# ak nemame nasledovatela odpoved je NIE 
if not nasledovatel: return NIE

# pockame kym bude vesmir ruzovy, ak niekto mal odpoved NIE, imperium nie je
# hamilotonovska kruznica, inak mozeme povedat ze odpoved je ANO
if ruzovy_vesmir(): return NIE
return ANO


\end{python}

\end{document}