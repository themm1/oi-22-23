\documentclass[10pt]{article}
\usepackage[a4paper,margin=1in,footskip=0.25in]{geometry}
\usepackage[final]{microtype}
\usepackage{fancyvrb}
\usepackage{xcolor}
\usepackage{amsmath}
\usepackage{pythonhighlight}

\title{\vspace{-3cm}Darček}

\vspace{0pt}
\begin{document}
\author{}
\date{}
\maketitle

\section*{Hlavná myšlienka}

V každom kroku sa číslo $c$ rozpadne na zvyšok a dve rovnaké čísla $x_1$ a
$x_2$. Keďže postupnosť vytvorená z čísla $x_1$ a $x_2$ je taká istá, stačí si
pamätať vzniknutý zvyšok a len jedno zo vzniknutých čísel ktoré budeme ďalej
týmto spôsobom rozkladať, až pokým sa vzniknuté $x_1$ a $x_2$ nebudú rovnať
nule.

\section*{Popis algoritmu}

Vznik postupnosti si môžeme predstaviť ako strom, kde výsledná postupnosť sú
zvyšky (prostredné deti každého vrcholu, ktorý nie je konečný, teda nie je
$0$). Ukážka stromu, kde $c=13$: (rovnakou farbou sú zvýraznené periodicky sa
opakujúce zvyšky)

\begin{Verbatim}[commandchars=\\\{\}]
                             13 
              6               \textcolor{red}{1}               6
      3       \textcolor{blue}{0}       3        \
      3       \textcolor{blue}{0}       3
  1   \textcolor{brown}{1}   1    \
  1   \textcolor{brown}{1}   1    \
  1   \textcolor{brown}{1}   1    \
  1   \textcolor{brown}{1}   1
0 \textcolor{orange}{1} 0  \
0 \textcolor{orange}{1} 0  \
0 \textcolor{orange}{1} 0  \
0 \textcolor{orange}{1} 0  \
0 \textcolor{orange}{1} 0  \
0 \textcolor{orange}{1} 0  \
0 \textcolor{orange}{1} 0  \
0 \textcolor{orange}{1} 0

\end{Verbatim}

Podľa spomenutej hlavnej myšlienky vieme, že si stačí pamätať stále len jedno
$j$. Budeme tak vytvárať len jednu vetvu stromu, počas čoho budeme počítať
výskyty aktuálneho zvyšku v danom intervale pomocou jeho periódy opakovania sa
vo výslednej postupnosti.

\subsection*{Dĺžka postupnosti vytvorenej z čísla $x$}

Postupnosť, ktorá vznikne z čísla $x$ je jednoznačne daná. Dĺžka tejto
postupnosti je počet zvyškov ktoré vzniknú v strome vzniku tejto postupnosti.
Môžeme si všimnúť, že v každej vrstve stromu sa počet vrcholov čo nie sú zvyšky
ani $0$, teda vrcholov v ktorých nekončí vetva, zdvojnásobí. Z každého z
takýchto vrcholov vznikne práve $1$ zvyšok. Keďže v každom kroku delíme $x$
dvomi až pokým $x$ nie je $0$, strom má $\lfloor log_2 x \rfloor + 2$ vrstiev.

Z predchádzajúcich tvrdení tak vyplíva, že počet zvyškov v strome vzniku
postupnosti je $2^{\lfloor log_2 x \rfloor + 2}$. To je tak aj dĺžka
postupnosti vytvorenej z čísla $x$. Na výpočet dĺžky tejto postupnosti z $x$ si 
môžeme definovať funkciu $f: y = 2^{\lfloor log_2 x \rfloor + 2}$. Pre účely
algoritmu si tiež definujme $f(0) = 0$, pretože ak $x=0$, vetvu, alebo celý
strom končíme. V špeciálnom prípade, keď dostaneme na vstupe $c=0$, to platiť
nebude, keďže výsledná postupnosť bude $0$ a teda má dĺžku $1$. To nás však
nezaujíma, pretože počet jednotiek v tejto postupnosti je $0$, tak isto ako
počet jednotiek v prázdnej postupnosti.

\subsection*{Perióda opakovania zvyšku vo výslednej postupnosti}

Zvyšky v strome majú ako stredné dieťa svojho predchodcu, ľavého a pravého
súrodenca. Medzi zvyškom $i$ a najbližším zvyškom z rovnakej vrstvy $j$ je tak
vždy:

\begin{itemize}
  \item Podstrom ľavého alebo pravého súrodenca zvyšku $i$.
  \item Zvyšok najbližšieho spoločného predchodcu týchto zvyškov. Keďže
  najbližší spoločný predchodca je vždy len jeden, aj tento zvyšok je vždy
  jeden. 
  \item Podstrom ľavého alebo pravého súrodenca zvyšku $j$.
\end{itemize}

Keďže veľkosť obidvoch podpostupností vytvorených z podstromov medzi $i$ a $j$
je $f(x)$, kde $x$ je súrodenec $i$, respektíve $j$, medzi $i$ a $j$ je
$2*f(x)+1$ zvyškov. Z toho vyplíva, že perióda zvyšku so súrodencami $x$ je
$2*f(x)+2$. Pre jednoduchosť označme $p: y = 2*f(x)+2$.

Prvá pozícia zvyšku $i$ vo výslednej postupnosti je tak $f(x)$, teda veľkosť
podstromu ľavého súrodenca. Vo vzťahu na periódu to tak môžeme zapísať ako
$p(x)/2-1$. Perióda je deliteľná dvomi, pretože $2*f(x)+2 = 2*(f(x) + 1)$.
Zvyšky z rovnakej vrstvy so súrodencom $x$ sa tak nachádzajú vo výslednej
postupnosti na indexoch $(p(x)/2-1)+k*p(x)$.

\subsection*{Spočítanie aktuálnych zvyškov nachádzajúcich sa v danom intervale}

Vieme, že ak sa na indexe $l$ nachádza zvyšok s aktuálnou periódou $i$, počet
týchto zvyškov v danom intervale bude $\lfloor(r - l) / i \rfloor + 1$. Pre
zvyšok je tak potretba vypočítať jeho prvý výskyt v danom intervale
$\langle l, r\rangle$ a dosadiť ho do spomenutého vzorca ako $l$.

Ak $(l - i / 2) \mod i \neq 0$, teda na indexe $l$ sa nenachádza aktuálny
zvyšok, musíme vypočítať $l_2$, ktoré bude značiť výskyt prvého aktuálneho
zvyšku v danom intervale. Vieme, že číslo $x$ môžeme zaokrúhliť na číslo $y$ 
nahor ako $(x + y) - (x + y) \mod y$. Všetky výskyty aktuálneho zvyšku sú
ale posunuté o $i/2$. Z toho vyplíva, že výsledný vzorec na posunutie $l$ bude
$(l + i) - (l + i - i / 2) \mod i = (l + i) - (l + i / 2) \mod i$. Takto vieme
vypočítať $l_2$, ktoré potom môžeme dosadiť do vzorca
$\lfloor(r - l) / i \rfloor + 1$ ako $l$, ak pôvodné $l$ nie je na pozícii
aktuálneho zvyšku.

V krajných prípadoch, kedy platí $l_2 > r$, nám stále výjde počet jednotiek
$0$, pretože $l_2 - r \leq i$, a keďže periódou $i$ aj delíme keď chceme zistiť
počet zvyškov na intervale výjde výsledok podielu $-1$ a $-1+1=0$. 

\subsection*{Celý algoritmus}

Počet jednotiek na intervale $\langle l, r \rangle$ tak môžeme vypočítať
nasledovným algoritmom: (na začiatku nastavíme počet jednotkových zvyškov
$s=0$)

\begin{itemize}
  \item Ak $c \mod 2 = 0$, môžeme aktuálny zvyšok preskočiť, keďže je $0$ a tak
  počet jednotiek vo výslednej postupnosti na intervale $\langle l, r \rangle$
  nijak neovplivní.
  \item Vypočítame periódu aktuálneho zvyšku ako $p(\lfloor c / 2 \rfloor)$.
  \item Spočítame aktuálne zvyšky nachádzajúce sa na danom intervale a ich
  počet pripočítame k celkovému počtu $s$.
  \item Zmeníme hondotu $c := \lfloor c / 2 \rfloor$ (tento krok vykonávame aj
  keď $c \mod 2 = 0$, teda aj keď periódu a súčet zvyškov v danom intervale
  nepočítame).
  \item Opakujeme pokým $c > 0$.
\end{itemize}

Výsledok, ktorý na konci vypíšeme je tak celkový počet jednotkových zvyškov
$s$.

\section*{Časová a priestorová zložitosť}

V algoritme delíme $c$ dvomi pokým $c > 0$. To znamená, že vždy vykonáme
rádovo $log_2 c$ krokov. Časová zložitosť je tak $O(log_2 c)$. Priestorová
zložitosť je $O(1)$. 

\section*{Python program}

\begin{python}
  
# vypocita dlzku postupnosti vytvorenej z cisla x
def size(x):
    if x == 0:
        return 0
    return 2 ** (floor(log2(x)) + 1) - 1

# vrati pocet jednotiek v postupnosti vytvorenej z c na intervale <l, r>
def solve(c, l, r):
    total_interval_sum = 0
    while c > 0:
        if c % 2 == 1:
            period = 2 * size(c // 2) + 2
            shifted_l = l
            if (l - period // 2) % period != 0:
                shifted_l = (l + period) - (l + period // 2) % period
            current_inteval_sum = (r - shifted_l) // period + 1
            total_interval_sum += current_inteval_sum
        c //= 2
    return total_interval_sum

input_line = [int(x) for x in input().split()]
answer = solve(*input_line)
print(answer)

\end{python}

\end{document}